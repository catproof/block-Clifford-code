\documentclass[rgb,listoffigures,listoftables,final]{cam-thesis}

% Packages go here

% Document Details
\title{An Efficient Quantum Message Authentication Scheme}
\author{Nicolas Perez}
\date{August 2022}
\thesistype{thesis}
\submittedto{the Faculty of Graduate and Postdoctoral Affairs}
\degree{Master}
\program{Computer Science}
\college{Carleton University}
\location{Ottawa, Ontario}
\submissiondate{2022}

% PDF meta-info:
\subjectline{\@title}
\keywords{keyword 1, keyword 2, etc.}

\abstract{%
Quantum networking is the study of transmitting data that uses quantum mechanics principles (quantum messages). This thesis describes an algorithm that detects when an adversary maliciously modifies a quantum message (an integrity attack). The algorithm is analyzed in terms of its ability to detect an integrity attack and its efficiency in terms of classical computational complexity and the number of quantum basic gates used. It is rigorously compared to the Clifford code quantum message authentication (QMA) scheme. It offers a trade-off compared to the Clifford code QMA scheme in terms of integrity attack detection versus efficiency: reduced detection rates and improved efficiency. The new algorithm is conjectured to satisfy a more lax alternative definition of QMA. As a consequence of developing this new algorithm, a novel method for unranking permutations of multisets is also presented.
}

\acknowledgements{
 I am grateful for the support from my supervisors Evangelos Kranakis and Michel Barbeau, and the additional support from Joaquin Garcia-Alfaro. Their inquiry and feedback on my research throughout my degree have been very helpful. My supervisors were also eager to teach me quantum computing at the start of my degree, and ensured that I had a good foundation of knowledge for starting my research.

My girlfriend provided me with an immense amount of support in more ways than I can say. She, along with my cats Steve and Bucky, were an appreciated source of comfort and humour and kept me good company during the isolating times of the COVID pandemic.

I used to ask my dad repeatedly while I was a small child: "What doing dad, what doing?", whilst he was tinkering around. He would let me watch and explain to me what he was doing. I used to ask my mom repeatedly: "Why? Why? Why?", when asking her about how the world worked. She would try her best to answer my questions! Both of them helped me become curious and want to know more about the world.

I am thankful for the support of all my family and friends.
}

%\setcounter{tocdepth}{1}